\documentclass[12pt]{article}



\usepackage{hyperref}

\usepackage{listings}
\usepackage[margin=1.25in]{geometry}
\usepackage{graphicx}
\graphicspath{ {./images/} }
\usepackage{imakeidx}
\makeindex[columns=3, title=Alphabetical Index, intoc]

\begin{document}

\begin{titlepage}

\title{%
  HW1: Mid-term assignment report\\
  \large  Testing and Software Quality\\}

\author{Rafael Remígio 102435}

\maketitle

\vfill
\begin{center}

	Departamento de Electrónica, Telecomunicações e Informática\\
       Universidade de Aveiro\\ Year 2022/2023
\end{center}



\end{titlepage}

\tableofcontents


\section{Introduction}

\subsection{Overview of the work} 

In this assignment, the web application developed provides real-time information parameters related to the air quality of a certain location.


\section{Product specification}


\subsection{Functional scope and supported interactions }


\subsubsection{Actors}

Possible Actors of the proposed product:

\begin{itemize}

\item Individuals - People who want to check the air quality in their immediate surroundings, such as their home, office, or neighborhood.

\item Public Health Officials - Officials who want to monitor air quality in a particular region or city to make informed decisions on public health policies.

\end{itemize}


\subsubsection{Use Cases}
The WebApp provides two use cases:

\begin{enumerate}

\item  Real-time Air Quality Monitoring. Provides real-time air quality levels in a particular location. This information consists of the specific parameters: 
	\begin{itemize}
		\item Particulate Matter PM2.5
		\item Particulate Matter PM10
		\item Carbon Monoxide CO
		\item Nitrogen Dioxide NO2
	\end{itemize}
	
\item access and view statistics of the cache's usage:
	\begin{itemize}
		\item Misses
		\item Hits
		\item Acesses
	\end{itemize}

\end{enumerate}


\subsection{System architecture}

\subsubsection{Architecture}
The application's Architecture is very basic and allows for easy maintenance and API evolution. The Controller handles communication between the WebApp and the backend. The Service Layer is composed of two distinct services. The Weather Service (responsible for providing the Air Quality for a certain region) utilizes a Geocoding API in order to transform an address or location string into coordinates. It communicates with two different Air Quality API's. In order to provide more constant availability and resilience, one acts as a backup when the main API is down. These API's provide Air Quality data for a certain location based on latitude and longitude.
In order to reduce query times, a caching system using a key-value database was also introduced. The objects cached have a specific Time to Live that is easily configurable.
\includegraphics[scale=.4]{architecture.png}

\subsubsection{Technologies and API's Used}

The very basic User Interface was constructed using ReactJS. It interacts with the Backedn througth a RestAPI. The backend was built using Spring Boot. For an in memory database I used Redis. The Geolocation API chosen was Open Cage. The primary Air Quality API is the OpenWeather API. The backup Air Quality API is the OpenMeteo API.

\includegraphics[scale=.4]{TechFrameworks.png}



\subsection{API}

The API has two endpoints:

\begin{enumerate}
	
	\item \textbf{GET} /weather?local=\textit{\{local\}} \\
	Query Params: 
	\begin{itemize}
			\item \textbf{Required} - local: Address, Country or City .
	\end{itemize}
		
	
	Response Body: \\ 
	\begin{lstlisting}
	{
	 "coords":{
	 	"lat":40.7275536,
	 	"lgn":-8.5209649
	 	},
	 "stats":{
	 	"co":150.0,
	 	"no2":6.5,
	 	"pm25":9.4,
	 	"pm10":11.5
	 	},
	 "location":"aveiro"
	}
	\end{lstlisting}
	
	
	\item \textbf{GET} /cacheInfo\\
	Response Body: \\
	\begin{lstlisting}
		{
		"hits":5,
		"misses":10,
		"acesses":15
		}
		
	\end{lstlisting}

\end{enumerate}

\section{Quality assurance}

\subsection{Logging}

Logging is a crucial aspect of software development that plays a vital role in supporting production and debugging activities. \\
Logging was conducted using the slf4j Logger class has it provides an easy integration with the Spring Boot Application.\\
Each log message contains a timestamp, method invocation details, custom log messages, and other contextual data. Additionally, each log entry is assigned a logging level identifier, providing a way to categorize and prioritize log entries based on their significance.\\
I followed logging principles and best practices - such as: logging at the correct level, providing meaningful messages, etc. - in order to provide useful logs.

\subsection{Overall strategy for testing}

The overall strategy used was for testing was Test-driven development (TDD). The idea behind TDD is to ensure that the code meets the requirements and is free from defects by writing tests that define the expected behavior of the code. 

\subsection{Unit and integration testing}

Unit Tests were used to test the individual units or components of the software application or system to ensure that they are functioning correctly. 
\begin{itemize}
\item Controller Tests: The goal of these tests is to ensure that the controller behaves correctly and produces the expected HTTP response. This involves creating mock objects for the multiple services that the controllers interact with.
\includegraphics[scale=0.5]{ControllerMocks.png}
\\
\includegraphics[scale=0.45]{ControllerExample.png}
\item Service Tests: The goal of these tests is to ensure that the Business Logic is implemented as expected. To achieve this, it is necessary to mock multiple components.
	\begin{enumerate}
		\item Mock the API calls
		\item Mock the Redis Database
		\item In the Weather Service tests it is also necessary to mock the Cache Service.
	\end{enumerate}
\includegraphics[scale=0.5]{ServiceMocks.png}
\includegraphics[scale=0.5]{ServiceMocks2.png}
\clearpage
Example of Mocks when \textit{"Location is not in cache but second API responds"}
\includegraphics[scale=0.5]{ServiceWhen.png}

\item Repository Tests: The goal of these tests is to ensure that the Database behaves as expected.

\includegraphics[scale=0.5]{CacheTest.png}

\item User Interface Testing - To automate tests of the Web Application, I used the Selenium WebDriver in conjunction with the Selenium IDE. 
\end{itemize}
\clearpage
\subsection{Code quality analysis}

For static code analysis, I used throughout the entirety of the project the SonarQube platform. It helped me identify potential bugs and vulnerabilities, improve code my code quality, and find code smells and possible antipatterns. In conjunction with static code analysis, I also used the SonarQube platform in order to evaluate the code coverage of tests implemented.

\includegraphics[scale=0.35]{sonarQube.png}
\clearpage
\subsubsection{Continuous Integration Pipeline}

A continuous integration (CI) pipeline is a process where code changes are automatically built, tested, and deployed to a target environment. GitHub Actions is a powerful platform for implementing a continuous integration pipeline. By defining the build, test, and deploy steps in the workflow file, the process of building, testing, and deploying your code changes can be automated.

\begin{center}
\includegraphics[scale=0.6]{CompleteIntegration.png}
\end{center}

\section{References \& resources}

All the code can be found and in my personal \href{https://github.com/Rafael-Remigio/TQS_102435/tree/main/HW1}{TQS repository}.\\
Videos of the working product can also be found on my personal TQS repository.

\end{document}