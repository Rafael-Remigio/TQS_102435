\documentclass[12pt]{article}



\usepackage{hyperref}

\usepackage[margin=1.25in]{geometry}
\usepackage{graphicx}
\graphicspath{ {./images/} }
\usepackage{imakeidx}
\makeindex[columns=3, title=Alphabetical Index, intoc]

\begin{document}

\begin{titlepage}

\title{%
  HW1: Mid-term assignment report\\
  \large  Testing and Software Quality\\}

\author{Rafael Remígio 102435}

\maketitle

\vfill
\begin{center}

	Departamento de Electrónica, Telecomunicações e Informática\\
       Universidade de Aveiro\\ Year 2022/2023
\end{center}



\end{titlepage}

\tableofcontents


\section{Introduction}

\subsection{Overview of the work} 


\subsection{Current limitations} 


\section{Product specification}


\subsection{Functional scope and supported interactions }

\subsection{System architecture}

\includegraphics[scale=.4]{architecture.png}

\subsection{API for developers}

\section{Quality assurance}

\subsection{Logging}

Logging is a crucial aspect of software development that plays a vital role in supporting production and debugging activities. 

Logging was conducted using the slf4j Logger class has it provides an easy integration with the Spring Boot Application.

Each log message contains a timestamp, method invocation details, custom log messages, and other contextual data. Additionally, each log entry is assigned a logging level identifier, providing a way to categorize and prioritize log entries based on their significance.

I followed logging principles and best practices - such as: logging at the correct evel, providing meaningfull messages, etc - in order to provide meaningfull logs.

\subsection{Overall strategy for testing}

\subsection{Unit and integration testing}

\subsection{Functional testing}

\subsection{Code quality analysis}

SonarQube

\subsection{Continuous integration pipeline}

\section{References \& resources}

GitRepository can be found at \"my github\".

\subsection{Reference materials}

\end{document}